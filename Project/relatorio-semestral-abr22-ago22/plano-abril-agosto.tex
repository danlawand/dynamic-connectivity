O primeiro passo estabelecido para o período de abril de 2022 até agosto de 2022, foi a criação de mais testes para fazer a análise de robustez e desempenho das \emph{link-cut trees}. As atualizações no código podem ser encontradas em: \texttt{https://github.com/danlawand/conexidade-dinamica}.

O segundo passo estabelecido foi de ao final do estudo das \emph{link-cut trees}, passar a estudar o problema mais geral, da conexidade totalmente dinâmica em grafos arbitrários.  Uma das soluções da literatura para este problema utiliza a biblioteca de conexidade para florestas~\cite{HolmLT2001}.

Trata-se de um algoritmo sofisticado, que mantém várias florestas geradoras do grafo corrente, representando níveis do grafo, e estas florestas são mantidas como no problema da conexidade dinâmica em florestas, por meio de \emph{link-cut trees}.  Projetávamos que esse estudo e a implementação deste algoritmo durasse cerca de 6 meses. 

Um terceiro passo seria a possibilidade de estudar as \emph{Euler tour trees}, uma outra estrutura de dados que pode ser usada de maneira semelhante às \emph{link-cut trees} na resolução do problema de conexidade totalmente dinâmica. Mas a prioridade é o estudo do problema geral. 

