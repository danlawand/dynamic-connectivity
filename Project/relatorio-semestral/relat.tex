\documentclass[12pt]{article}
\usepackage[utf8]{inputenc}
\usepackage[brazil]{babel}
\usepackage{import}
\usepackage{verbatim}
\usepackage{graphicx}
\usepackage{fullpage}

\usepackage{listings}
\usepackage{color}

\definecolor{dkgreen}{rgb}{0,0.6,0}
\definecolor{gray}{rgb}{0.5,0.5,0.5}
\definecolor{mauve}{rgb}{0.58,0,0.82}

\usepackage{floatflt,epsfig,epsf}
\usepackage[dvipsnames]{xcolor}

\lstset{
  language=C,
  basicstyle=\footnotesize,
  numbers=left,
  numberstyle=\tiny\color{gray},
  stepnumber=1,
  numbersep=5pt,
  backgroundcolor=\color{white},
  showspaces=false,
  showstringspaces=false,
  showtabs=false,
  frame=none,
  tabsize=2,
  captionpos=b,
  breaklines=true,
  breakatwhitespace=false,
  title=\lstname,
  keywordstyle=\color{blue},
  commentstyle=\color{dkgreen},
  stringstyle=\color{mauve},
}

\newtheorem{defini}{Definição}[section]
\newtheorem{prob}{Problema}[section]
\newcommand{\Oh}{\mathrm{O}}

\sloppy

\begin{document}
\begin{center}

{\Large {\bf Conexidade Dinâmica}} 

\vspace{0.2cm}
{\large {\em Relatório de Atividades da Iniciação Científica}
}

\vspace{0.2cm}
{\small 
{\bf Orientadora:} Cristina Gomes Fernandes \\
{\bf Aluno:} Daniel Angelo Esteves Lawand
}

\vspace*{\fill}
{\small Este relatório refere-se à bolsa de Iniciação Científica \\
para o aluno Daniel Angelo Esteves Lawand,\\ 
e cobre o período de setembro de 2021 a março de 2022.}

\end{center}
\vspace*{\fill}

\newpage

\section{Introdução}  
\import{./}{introducao.tex}

\section{Splay trees}  
\import{./}{splay.tex}

\section{Link-cut trees}  
\import{./}{lct.tex}

\section{Conexidade totalmente dinâmica em florestas}  
\import{./}{floresta-dinamica.tex}

\section{Planos para o próximo período}  
\import{./}{conclusao.tex}

\bibliographystyle{plain}
\bibliography{relat}

\end{document}
