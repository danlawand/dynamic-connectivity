O próximo passo será a criação de mais testes para fazer a análise de robustez e desempenho das \emph{link-cut trees}. Se encontrarmos resultados interessantes, registraremos essas medidas no repositório em que o código fonte está. Este pode ser encontrado em: \texttt{https://github.com/danlawand/conexidade-dinamica}.  

Ao final desse estudo das \emph{link-cut trees}, passaremos a estudar o problema mais geral, da conexidade totalmente dinâmica em grafos arbitrários.  Uma das soluções da literatura para este problema utiliza a biblioteca de conexidade para florestas~\cite{HolmLT2001}. 
Trata-se de um algoritmo sofisticado, que mantém várias florestas geradoras do grafo corrente, representando níveis do grafo, e estas florestas são mantidas como no problema da conexidade dinâmica em florestas, por meio de \emph{link-cut trees}.  Projetamos que esse estudo e a implementação deste algoritmo dure cerca de 6 meses. 

Outro tópico relacionado que poderia ser estudado são as \emph{Euler tour trees}, uma outra estrutura de dados que pode ser usada de maneira semelhante às \emph{link-cut trees} na resolução do problema de conexidade totalmente dinâmica.  Mas priorizaremos o estudo do problema geral no próximo período. 

